\documentclass[10pt,a4j,twocolumn]{jsarticle}
\usepackage{csbstp}

\usepackage[dvipdfmx]{graphicx}
\graphicspath{{./img/}}
\usepackage{amsmath}
\usepackage{amssymb}
\usepackage{multirow}
\usepackage{url}
\usepackage{here}
\usepackage{tabularx}
\usepackage{longtable}
\newcommand{\jump}[1]{\ensuremath{[\![#1]\!]} }
\renewcommand{\figurename}{Fig. }
\renewcommand{\tablename}{Table }

% define argmax & argmin
\DeclareMathOperator*{\argmin}{arg\,min}
\DeclareMathOperator*{\argmax}{arg\,max}

\title{Your Thesis Title}
\author{Your Name}
\studentid{Student ID}
\university{早稲田大学}%省略可
\faculty{基幹理工学部}%省略可
\department{\footnotesize Computer~Science\\and~Communications\\Engineering}%省略可
\guidance{~Information\\Access}
\type{修士論文}%省略可
\nendo{202X}
\hizuke{mm/dd/202X}
\advisor{Tetsuya Sakai}

\begin{document}
\setlength{\abovedisplayskip}{0pt}
\setlength{\belowdisplayskip}{0pt}
\renewcommand{\baselinestretch}{0.9}
\maketitle

\section{Abstract}
Write your abstract here.\cite{RSL}

\section{Proposed Methods}

\subsection{\rmfamily blablabla}

\section{Experimental Results and Discussions}

\section{Conclusions}

\small
\bibliographystyle{abbrv}
\bibliography{abs}

\end{document}
